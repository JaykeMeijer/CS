\documentclass[a4paper]{article}
\usepackage[english]{babel}
\usepackage[utf8]{inputenc}
\usepackage{alltt,amsmath,hyperref,graphicx}
\usepackage[usenames,dvipsnames]{xcolor}

% Link colors
\hypersetup{colorlinks=true,linkcolor=black,urlcolor=blue,citecolor=OliveGreen}

% Set paragraph indentation
\parindent 0pt
\parskip 1.0ex plus 0.5ex minus 0.2ex

\title{Internet Programming - Assignment 4}
\author{Jayke Meijer, 2526284, jmr251, \url{jayke.meijer@gmail.com}}

\begin{document}

\maketitle

\tableofcontents
\pagebreak

\section{Used platform}

All code is tested and executed on a machine running Ubuntu 12.04. The used version of
\emph{gcc} is 4.6.3. The JDK used is OpenJDK IcedTea6 1.11.4. The Java version is
1.6.0\_24. The used version of Apache is 2.2.23, the used version of PHP is 5.4.8.

\section{Paper Storage}

The first part of this assignment is to implement a webpage for the paper storage system, using
CGI. Since the questions ask about some vital parts of the design, these will be answered 
first.

\subsection{Questions}

\textbf{Question A} The CGIs are implemented in C. This is done since the paper server itself is in C,
and it uses Sun RPC. By using C, I can directly use the RPC functions to communicate to the server,
eliminating the need for any explicit middleware that translates the requests (this is handled by Sun RPC).

\textbf{Question B} Because the paper server is designed to work with RPC. If we were to use PHP, we would
first need to translate the request from PHP to an RPC call, adding another step. By using CGI, we can
write the server in any language, including C, which makes it easy to perform RPC.

\textbf{Question C} I designed an upload such as described in rfc1867. The data is send by using POST,
so the size of the data is stored in Environment variables. The data itself is received from stdin, which
is possible since the size is known. The data is then parsed, by reading the lines on which the data is
stored, one for the author, one for the title and a variable number of lines for the actual paper. This is
copied to local memory, to be added to the server (using RPC).

An alternative is to upload the paper using PHP, and have the CGI read it from the upload location. However,
this is not conform the assignment.

\subsection{Further design}

To print a HTML page, a shared C file called \texttt{html.c} is created, which contains functions to print
partial html code that is used in several locations. This allows for cleaner code, since strings with
html code can be rather large.

The CGI to print the papers in the system receives a linked list with data on each paper. It makes a new
list item for each of these papers.

The CGI to view a paper writes the header (since the data type is not stored on the server, we assume we only
work with PDF's). It then writes the binary data to the client.

\section{Hotel reservations}

The second part of the exercise is to build a webpage for the hotel reservation system. There are two versions
required, one using only PHP, the other using AJAX.

\subsection{PHP}

The first version uses only PHP. The entire page is build at the server, and send as HTML to the client.
The page is static, and refreshing the data requires reloading the page.

The communication with the hotel server is performed using sockets and the hotel gateway application.

For reserving a room, a form is filled in and the server uses this data to actually book the room. For listing
the free rooms or the guests, the data is requested over the socket, received and unpacked. It arrives in a
string with separation characters (either \texttt{$\backslash$t} or \texttt{$\backslash$n}). It is unpacked using \texttt{explode()}.

\subsection{AJAX}

The AJAX version does not use PHP to build the page. The page is static HTML (and some CSS) and uses JavaScript
to dynamically include the variable data. To do so, an AJAX request is made to the server. On the server, this
request is handled using PHP, which replies using JSON.

The JSON reply is used by the JavaScript to create and place the data in the DOM tree, and therefore on the page.

A special function is the function to reload the page. This requests the data again from the server, and rebuilds
the lists. By calling the function \texttt{rebuildList()} and \texttt{rebuildGuests} from a timer, it is possible
to create a page that updates on a given interval.

The booking of room is done from a popup (a DOM object that is hidden or not). On this popup there are some form
elements, making it look like a form. However, the button does not perform a form submit, but calls a JavaScript
function that places an AJAX call with the data filled in on the form.

\subsection{Questions}

\textbf{Question D} It would have been possible to use a CGI. If you were to use it directly on the server,
bypassing the gateway, this CGI would have to be written in Java. Starting a JVM for every call is
costly, so in that case it would be advisable to use servlets.

The AJAX version is also possible to implement using CGI or servlets. The AJAX call can be made to a CGI or servlet,
as long as the program responds using JSON.

\end{document}
