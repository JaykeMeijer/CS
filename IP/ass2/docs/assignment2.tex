\documentclass[a4paper]{article}
\usepackage[english]{babel}
\usepackage[utf8]{inputenc}
\usepackage{alltt,amsmath,hyperref,graphicx}
\usepackage[usenames,dvipsnames]{xcolor}

% Link colors
\hypersetup{colorlinks=true,linkcolor=black,urlcolor=blue,citecolor=OliveGreen}

% Set paragraph indentation
\parindent 0pt
\parskip 1.0ex plus 0.5ex minus 0.2ex

\title{Internet Programming - Assignment 2}
\author{Jayke Meijer, 2526284, jmr251, \url{jayke.meijer@gmail.com}}

\begin{document}

\maketitle

\tableofcontents
\pagebreak

\section{Used platform}

All code is tested and executed on a machine running Ubuntu 12.04. The used version of
\emph{gcc} is 4.6.3. The JDK used is OpenJDK IcedTea6 1.11.4. The Java version is
1.6.0\_24.

\section{A Content-preserving Server}

\subsection{Part 1}

The first task is to write an iterative TCP server, that maintains a counter and outputs
the value of this counter to a connecting client. Then it closes the connection and
increments the counter.

This is all iterative. This means that the first program waits for a connecting client. 
Once a client has connected, the server sends the value of the counter and internally 
increases it. After this is done, it will wait for a new connection.

\subsection{Part 2}

The second part of the exercise is to create a server that is implemented according to the
one-process-per-request principle. This means the server waits for a new connection. Once
this connection is established it forks to create a child process. This child process
handles the actual writing of the counter, while the parent can immediately wait for a new
incoming connection.

\textbf{Question A} The problem is that the counter has to be maintained properly. It must
be guaranteed that each client connection receives a value that is exactly one higher than
the previous connection. To solve this, the parent should increase the counter between two
accept calls. This way the server will behave correctly when there are multiple concurrent
requests, since the counter is already increased before accepting a connection, and only
one connection can be accepted at a time (although the further handling of these
calls can be done in parallel).

The implementation does exactly this. Once a connection is accepted the process forks. The
parent than immediately closes the socket connection to the client, increases its counter
and only then returns to accepting a new client.

Meanwhile, the child process will write the counter to the client. Note that this counter 
still has the same value that it had at the time of accepting this client, since the
increasing of the counter was done in the parent process, not in the child.

\subsection{Part 3}

The third server to implement is a preforked server. A preforked server will initiate a
certain number of child processes, which will then in turn handle the incoming 
connections.

\textbf{Question B} There is a number of issues to solve. The first is that each child
should be able to accept connections that arrive on the same listening socket. The second
problem is that the counter has to be shared between these child processes, since each
client should be guaranteed to receive a value that is exactly one higher than the
previous client.

The solution to the first problem is already in the \texttt{accept()} function. This will
remove the connection request once called, meaning only the first process to execute
\texttt{accept()} will actually receive a connection. However, this means that if there
are N processes, N-1 processes will be woken up for nothing by an incoming connection.
To solve this the accept call should be made mutually exclusive. This means only one 
child will be woken up by an incoming connection, namely the child that is in the
critical section.

The second problem is the counter. Since there are different processes, a way has to be
found to share the current value of the counter between these processes. For this, shared
memory can be used. This is a memory segment that is shared between several processes.

There is a problem with shared memory though: you have to be careful with race conditions. 
Because incrementing the counter is not an atomic operation, there can be
problems with the counter being read by a second program before it is raised by the first.
To tackle this problem the retrieving of the counter and increasing its value have to be
mutually exclusive. This means that a lock mechanism has to be placed around this section.

The server will now behave correctly, since the counter is in shared memory and is now 
guaranteed to be increased every time it is read, and it is not read by any other process
between reading and increasing by the first process.

The implementation now has a challenge, since there is a piece of shared memory and two
semaphores (used as locks for the incrementing of the counter and the \texttt{accept()}
call. These will continue to exist even when all the processes using them have
stopped. Therefore the program needs to explicitly destroy these shared objects. To do 
this, the signal \texttt{SIGINT} is caught. \texttt{SIGINT} is the signal generated by
\emph{ctrl + C}. When this signal is received, the parent process will kill the child
processes by sending \texttt{SIGTERM} to each of them and will then destroy the shared 
memory section and the two semaphores. Alternatively, \emph{exit} can be written to 
\texttt{stdin}, in which case the same actions will be performed.

\section{A Talk Program}

The second assignment is the creation of a talk program. Two users should be able to
communicate over a TCP connection.

\subsection{Part 1}

The first part is establishing a connection. This is a challenge since the program has to
be able to have a different behavior depending whether it is the server or the client.
After that it has to perform identically.

To be able to do this, the program checks whether an address is given as argument. If so,
the program will behave as a client and will create a connection to the given address.
Otherwise, the program will behave like a server, and set up a listening socket. Once a
connection is established, this listening socket will be closed and only the connection
remains. From this point out, the programs behave identical in a peer-to-peer like manner.

\textbf{Question C} The most appropriate server type in this case is an iterative server.
Since there is only one connection, which only has to be established once, any other type
of server would basically do exactly the same job, only with the overhead of at least one
fork.

\textbf{Question D} A client and a server on the same computer is no problem, since these
both create a socket, which can be connected to localhost. Two servers on one machine is
not directly possible, since there is no immediate way to separate the messages coming in.
The solution is to use different ports. In the code a port is hardcoded. However, it would
be possible to have this port number as an argument to the program, allowing each server
instance to use its own port (and of course the client has to connect to the corresponding
port).

\subsection{Part 2}

The second challenge is how to be able to concurrently be able to receive both messages
over the socket and wait for input on \texttt{stdin}. Both \texttt{read()} commands on
these are blocking. This would mean they can only be used one after another.

\textbf{Question E} There are two possible solutions to this problem. The first is to make
the program multithreaded. In that case the program will have two threads, one for the
receiving and one for the sending of data (and getting that data from the user). The other
option is to use \texttt{select()}. Select makes it possible to ``multiplex'' between
several inputs in a single process.

The solution chosen for this program is the first, multithreading. The reason is that this
properly utilizes the potential of a machine, by using two processor cores if available.
One of the downside of threads is scalability, where a large number of threads would mean
a lot of thread switches in the CPU. This is a serious problem in server applications that
create a thread for each connection. However, since this program only uses two threads, 
one for receiving and one for writing, this is not a problem.

Another disadvantage of threads is the need for synchronization. In this application 
though, the only shared resource that can create a race condition is \texttt{stdout}. The
assignment explicitly states that this is not something to be concerned with. Therefore, 
the extra programming effort needed is minimal, whereas getting select to work properly
requires quite some extra coding.

This only leaves the need to properly close the connections and both threads, even when
these are in the blocking \texttt{read()} function. For this, signals between the threads
are used. There are two cases that lead to shutdown: the socket is closed or the user 
types \emph{!exit}. The first case is noticed in the `child' thread. This thread will then
signal to the parent that it is shutting down and exit. The `parent' receives this signal,
waits for the other thread to join and will exit, ending the program.

The other case is noticed in the parent itself. It will signal the other thread to end, 
and wait for this thread to end, before ending itself.

\end{document}