\documentclass[a4paper]{article}
\usepackage[english]{babel}
\usepackage{alltt}
\usepackage{amsmath}
\usepackage{url}
\usepackage{graphicx}

\title{Concurrency and Multithreading - Design}

\author{Jayke Meijer, 2526284, \url{jayke.meijer@gmail.com} \\
Tadde\"us Kroes, , \url{}}

\begin{document}

\maketitle
\pagebreak

\tableofcontents
\pagebreak

\section{Introduction}

This document discusses the design of several concurrent data
structures. Both an advanced design and a discussion about the
expected performance will be given for each data structure.

The data structures are the following: CoarseGrainedList,
CoarseGrainedTree, FineGrainedList, FineGrainedTree,
LockFreeList and LockFreeTree.

The difficulty with these data structures is that they are
accessed by several threads concurrently, so mutual exclusion has
to be provided for the parts where the updates take place.

\section{CoarseGrainedList}

The first data structure discussed is the CoarseGrainedList.
This is a singly-linked list. When an element is added or removed,
the entire data structure is locked.

\subsection{Design}

For the CoarseGrainedList a singly-linked list is required. To create this
list a sequence of nodes is used. These nodes contain both the stored items
and a link to the next node in the list.

To lock the list when an \emph{add} or \emph{remove} operation is performed,
a single lock is required. The lock is acquired in the beginning of the
operation. Once the lock is obtained, either the position to insert the new
node or the node to remove is located, and the remove or insert operation is
performed. Once this is successfully done, the lock is released, allowing
other threads to do their operations.

\subsection{Performance}

The performance of the CoarseGrainedList is expected to be very poor. Almost
no parallel execution is possible, due to the fact that the entire
data structure is locked during the entire add or remove operation. This means
that whenever one thread is performing such an operation, no other thread can
do such an operation, but also not even prepare this operation, by for example
locating the nodes affected.

This means that regardless of the number of elements in the data structure,
only one thread at a time can access the list. The performance is therefore
expected to be very poor when a lot of threads are trying to execute a lot
of operations on the list.

The complexity for a single thread will be of $\mathcal{O}(n)$ with $n$ being
the number of elements in the list, since
locating the affected nodes means in worst case looping through the entire
list.

Since no parallel execution of the operations is possible, the worst-case
complexity of the entire algorithm is expected to be $\mathcal{O}(n * m)$,
with $n$ the number of elements and $m$ the number of threads.

\section{CoarseGrainedTree}

The CoarseGrainedTree is a binary search tree. As with the
CoarseGrainedList, the complete data structure is locked when
a add or remove is performed.

\subsection{Design}

The design of the CoarseGrainedTree is comparable with the design of the
CoarseGrainedList. However, the nodes used now have up to two child nodes
instead of a single following node.

The locking mechanism remains the same, the entire tree is locked before
an add or remove operation is performed. Again, the operation performed
during the lock includes the locating of the affected nodes.

\subsection{Performance}

As with the CoarseGrainedList, the performance will be poor when a lot of
threads try to perform operations simultaneously.

The locating of the affected nodes will be faster, since we are dealing
with a binary tree instead of a list. Therefore the overall performance
will be better, due to the fact that the basic algorithm is more effective.

The complexity of a single thread is in the worst-case the same as with the
linked list, when the tree is completely unbalanced, so $\mathcal{O}(n)$.
However, the average complexity will be better than that, and better than the
average complexity of the linked list.

If the tree is completely balanced, the complexity will be of order
$\mathcal{O}(log n)$.

Once again, these complexities are to be multiplied with the number of
threads, giving $\mathcal{O}(n * m)$ and $\mathcal{O}(log(n) * m)$
respectively.

\section{FineGrainedList}

The FineGrainedList is a linked list that only locks a part of
the data structure when performing an add or remove operation. This
gives it the ability to have several threads update the list at the
same time.

\subsection{Design}

Design of FGL.

\subsection{Performance}

Performance Evaluation of FGL.

\section{FineGrainedTree}

The FineGrainedTree is a tree in which only a part of the data structure
will be locked for add or remove operations, as with the FineGrainedList.

\subsection{Design}

Design of FGT.

\subsection{Performance}

Performance Evaluation of FGT.

\section{LockFreeList}

The LockFreeList is a linked list that does not use a lock when updating
the datastructure. Instead it uses a single atomic operation for both
comparing and updating, compare-and-set instructions.

\subsection{Design}

Design of LFL.

\subsection{Performance}

Performance Evaluation of LFL.

\section{LockFreeTree}

The LockFreeTree uses compare-and-set instructions in stead of locks.

\subsection{Design}

Design of LFT.

\subsection{Performance}

Performance Evaluation of LFT.

\end{document}