\documentclass[a4paper]{article}
\usepackage[english]{babel}
\usepackage[utf8]{inputenc}
\usepackage{alltt,amsmath,hyperref,graphicx}
\usepackage[usenames,dvipsnames]{xcolor}

% Link colors
\hypersetup{colorlinks=true,linkcolor=black,urlcolor=blue,citecolor=OliveGreen}

% Set paragraph indentation
\parindent 0pt
\parskip 1.0ex plus 0.5ex minus 0.2ex

\title{Concurrency and Multithreading - Performance Evaluation}
\author{Jayke Meijer, 2526284, jmr251, \url{jayke.meijer@gmail.com} \\
Taddeüs Kroes, 2526902, tks590, \url{taddeuskroes@gmail.com}}

\def\bigoh{\mathcal{O}}

\begin{document}

\maketitle

\tableofcontents
\pagebreak

\section{Introduction}

This document discusses the design and performance of several
concurrent data structures.

The data structures are the following: CoarseGrainedList,
CoarseGrainedTree, FineGrainedList, FineGrainedTree,
LockFreeList and LockFreeTree.

The difficulty with these data structures is that they are accessed by several threads 
concurrently, so it has to be guaranteed that each update succeeds, even when separate
updates by different threads interfere.

The design of the data structures was already given in the design document\cite{design}. 
These will therefore only be discussed on a high level in this performance evaluation. Any
difficulties encountered during the implementation will be discussed.

\section{Implementation difficulties}

\subsection{Lists and CoarseGrainedTree}
Most of the data structures were straightforward to implement. The CoarseGrainedList,
FineGrainedList and LockFreeList were described in the book by Herlihy and 
Shavit\cite{book}. These did therefore not present any problems during the implementation.

The CoarseGrainedTree had to be designed by ourself, but only required the placing of a 
lock mechanism around standard tree insertions and removals. The used algorithms for 
inserting and removing in a Binary Search Tree are based on standard BST operations.
\cite{insert}\cite{delete}

\subsection{FineGrainedTree}

The FineGrainedTree is more of a challenge to implement. The basic idea is the same as the
FineGrainedList. It uses hand-over-hand locking to traverse through the list. However,
instead of locking the successor you first have to determine whether to lock the left or
the right child node.

The challenge is what to do during the removal of an element. When deleting a node that
has two children, the in-order successor of this node has to be found. This is the node
with the smallest value in the right subtree of the node to delete. However, it is 
possible that another thread is performing an operation in this subtree. Therefore 
traversal through this subtree should be done with locking as well, to prevent overtaking
of another thread.

\subsection{LockFreeTree}

The most challenging data structure is the LockFreeTree. The design is provided in an 
article by Ellen, Ruppert, Fatourou and Van Breugel.\cite{lft} This article describes the
design, gives a pseudocode implementation and proves why it is correct.

The first challenge encountered is that the article uses Compare-And-Swap instructions.
However, Java does have this operation, it only supports Compare-And-Set instructions. The
difference between these two is that a Compare-And-Swap returns the original value, even
when the operation failed. The Compare-And-Set only returns whether the operation 
succeeded or not. %% Invullen waarom dat geen probleem is, met name HelpDelete()

Another challenge is that the article stores the \texttt{Update} datatype in a single CAS
word, and uses a part of this word for the flag (only 2 bits) and the rest for the info
pointer

\section{Performance Evaluation}

\subsection{Used hardware}

\section{Conclusion}

\begin{thebibliography}{}    
\bibitem{design}
    Taddeüs Kroes, Jayke Meijer, Concurrency and Multithreading - Design

\bibitem{book}
    Maurice Herlihy \& Nir Shavit, The Art of Multiprocessor Programming. Morgan Kaufmann.         
    2012.

\bibitem{insert}
    Binary Search Tree: Insertion
    \url{http://www.algolist.net/Data_structures/Binary_search_tree/Insertion}
    
\bibitem{delete}
    Binary Search Tree: Removal
    \url{http://www.algolist.net/Data_structures/Binary_search_tree/Removal}

\bibitem{lft}
    Faith Ellen, Eric Ruppert, Panagiota Fatourou, Franck van Breugel, Non-blocking Search 
    Trees

\end{thebibliography}

\end{document}